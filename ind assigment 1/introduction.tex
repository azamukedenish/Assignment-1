\chapter{Introductory background }

Various dynamics are currently changing the structure of the poultry sector. In 1999, Delgado et al. (1999) labeled the massive changes taking place in the livestock sector the "livestock revolution". The label covers the complex of trends, processes and effects that characterizes global livestock demand and supply. In brief, the growth in global demand for meat and other livestock products is tremendous - fuelled by population growth, economic growth, urbanization, changing diets and reductions in the relative prices of livestock products
The poultry industry faces the challenge of trade barriers that have been implemented in some countries such as China and India. Russia and India have introduced policies in their respective countries meant to hamper entry of the US poultry products (McArdle, 2006). In Malaysia, it is the high feed cost and new emerging diseases as the main challenges in the poultry industry (Razak, 2011), while in India the problems faced by industry are: Feed cost, Ignorance regarding biosecurity and egg and broiler highly fluctuating rates (Dr. Narayan, 2011)

\subsection{Key assumptions to the study}

\subsubsection{Assumptions of the study}
The research study was based on following assumptions: That all respondents would be cooperative and provide reliable responses through the questionnaires distributed and within the given time. It was based on the belief that the farmers selected were motivated by profits from the poultry projects. It was assumed that sampled, sub population would be a true representative of the population within Omugo sub county.



